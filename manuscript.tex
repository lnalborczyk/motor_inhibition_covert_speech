%%%%%%%%%%%%%%%%%%%%%%%%%%%%%%%%%%%%%%%%%%%%%%%%%%%%%%%%%%%%%%%%%%%%%%%%%%%%%%%%%%%%%%%%%%%%%%%%%%%%%%%%%%%%%%%%%%%%%%%%%%%%%%%%%%%%%%%%%%%%%%%%%%%%%%%%%%%
% This is just an example/guide for you to refer to when submitting manuscripts to Frontiers, it is not mandatory to use Frontiers .cls files nor frontiers.tex  %
% This will only generate the Manuscript, the final article will be typeset by Frontiers after acceptance.   
%                                              %
%                                                                                                                                                         %
% When submitting your files, remember to upload this *tex file, the pdf generated with it, the *bib file (if bibliography is not within the *tex) and all the figures.
%%%%%%%%%%%%%%%%%%%%%%%%%%%%%%%%%%%%%%%%%%%%%%%%%%%%%%%%%%%%%%%%%%%%%%%%%%%%%%%%%%%%%%%%%%%%%%%%%%%%%%%%%%%%%%%%%%%%%%%%%%%%%%%%%%%%%%%%%%%%%%%%%%%%%%%%%%%

%%% Version 3.4 Generated 2018/06/15 %%%
%%% You will need to have the following packages installed: datetime, fmtcount, etoolbox, fcprefix, which are normally inlcuded in WinEdt. %%%
%%% In http://www.ctan.org/ you can find the packages and how to install them, if necessary. %%%
%%%  NB logo1.jpg is required in the path in order to correctly compile front page header %%%

\documentclass[utf8]{template/frontiersSCNS} % for Science, Engineering and Humanities and Social Sciences articles
%\documentclass[utf8]{frontiersHLTH} % for Health articles
%\documentclass[utf8]{frontiersFPHY} % for Physics and Applied Mathematics and Statistics articles

%\setcitestyle{square} % for Physics and Applied Mathematics and Statistics articles
% \usepackage{url,hyperref,lineno,microtype,subcaption}
\usepackage{url,hyperref,lineno,microtype}
\usepackage[onehalfspacing]{setspace}

%%%%%%%%%%%%%%%%%%%%%%%%%%%%%%%%%%%%%%
% Create explanation Box environment %
%%%%%%%%%%%%%%%%%%%%%%%%%%%%%%%%%%%%%%

\usepackage{tcolorbox}
\tcbuselibrary{skins,breakable}

\newtcolorbox[auto counter]{mybox}[2][]{%
    enhanced,
    fonttitle = \bfseries,
    title = Box~1: #2,
    #1
    }

% commented out when sending the abstract
\linenumbers

% Leave a blank line between paragraphs instead of using \\

\def\keyFont{\fontsize{8}{11}\helveticabold }
\def\firstAuthorLast{Nalborczyk {et~al.}} %use et al only if is more than 1 author
\def\Authors{Ladislas Nalborczyk\,$^{1,2,*}$, Ursula Debarnot\,$^{3,4}$, Marieke Longcamp\,$^{2}$, Aymeric Guillot\,$^{3,4}$, and F.-Xavier Alario\,$^{1}$}
% Affiliations should be keyed to the author's name with superscript numbers and be listed as follows: Laboratory, Institute, Department, Organization, City, State abbreviation (USA, Canada, Australia), and Country (without detailed address information such as city zip codes or street names).
% If one of the authors has a change of address, list the new address below the correspondence details using a superscript symbol and use the same symbol to indicate the author in the author list.
\def\Address{$^{1}$Aix Marseille Univ, CNRS, LPC, Marseille, France \\
$^{2}$Aix Marseille Univ, CNRS, LNC, Marseille, France \\
$^{3}$Inter-University Laboratory of Human Movement Biology-EA 7424, University of Lyon, University Claude Bernard Lyon 1, Villeurbanne, France \\
$^{4}$Institut Universitaire de France, Paris, France}
% The Corresponding Author should be marked with an asterisk
% Provide the exact contact address (this time including street name and city zip code) and email of the corresponding author
\def\corrAuthor{Corresponding Author}
\def\corrEmail{ladislas.nalborczyk@univ-amu.fr}

\begin{document}
\onecolumn
\firstpage{1}

\title[Motor inhibition and covert speech]{The role of motor inhibition during covert speech production}

% Alternative title: Does covert speech production imply the motor inhibition of overt speech acts?

\author[\firstAuthorLast ]{\Authors} %This field will be automatically populated
\address{} %This field will be automatically populated
\correspondance{} %This field will be automatically populated

\extraAuth{}% If there are more than 1 corresponding author, comment this line and uncomment the next one.
%\extraAuth{corresponding Author2 \\ Laboratory X2, Institute X2, Department X2, Organization X2, Street X2, City X2 , State XX2 (only USA, Canada and Australia), Zip Code2, X2 Country X2, email2@uni2.edu}

\maketitle

\begin{abstract}

\noindent Covert speech is accompanied by a subjective multisensory experience with auditory and kinaesthetic components. An influential hypothesis states that these sensory percepts result from a simulation of the corresponding motor action that relies on the same internal models recruited for the control of overt speech. A significant consequence of this hypothesis is that the sensory experience of covert speech would be continuously shaped by sensorimotor interactions with the environment. However, the precise computations required by these internal models and their neural implementation remain unclear. Moreover, this simulationnist view raises the question of how it is possible to imagine actions without executing them. In this perspective, we focus on covert speech as an exciting case study to help understanding the interplay between language, motor control, and perceptual processes. 

% Since the first explorations of the phenomenological and psychophysiological properties of imagined actions, there has been considerable efforts and progresses towards describing the mechanisms leading to these sensory percepts.

\tiny
 \keyFont{\section{Keywords:} covert speech, inner speech, motor imagery, motor simulation, motor control, response inhibition} % All article types: you may provide up to 8 keywords; at least 5 are mandatory.
\end{abstract}

\newpage

\section{Introduction}

The ability to mentally examine our thoughts is central to our subjective experience. The internal (covert) production of speech may accompany common activities such as problem solving \citep{baldo_is_2005, sokolov_inner_1972}, future planning \citep{dargembeau_frequency_2011}, reading \citep[e.g.,][]{loevenbruck_left_2005, perrone-bertolotti_how_2012}, or writing \citep{frith_reading_1979}. Because speech production results from sequences of motor commands that are assembled to reach a given communication goal, it belongs to the broader category of motor actions \citep{jeannerod_motor_2006}. Therefore, a parallel can be drawn between covert speech, also known as \textit{inner speech} or \textit{speech imagery} \citep[for reviews, see][]{alderson-day_inner_2015, perrone-bertolotti_what_2014, loevenbruck_cognitive_2018}, and other forms of covert verbal actions such as covert writing or covert typing and, more generally, other imagined actions (motor imagery).

% One of the most influential theoretical model of motor imagery the motor simulation theory \cite{jeannerod_origin_2006, jeannerod_representing_1994, jeannerod_neural_2001}, postulates the existence of a continuum between the covert and the overt execution of an action and that action representations can operate off-line via a simulation mechanism. In short, covert actions would rely on the same set of mechanisms as the overt actions they simulate, except that execution is inhibited \citep{oshea_does_2017}. A second class of models are concerned with the phenomenon of emulation \citep{grush_emulation_2004} and with internal models \cite[for a review of the similarities and dissimilarities between simulation and emulation models, see][]{gentsch_towards_2016}. Internal model theories share the postulate that action control uses internal models, that is, systems that simulate the behaviour of the motor apparatus \citep[e.g.,][]{jordan_forward_1992, kawato_hierarchical_1987}. The function of internal models is to estimate and anticipate the outcome of a motor command. Among the internal model theories, motor control models based on robotic principles \citep[e.g.,][]{kawato_internal_1999, wolpert_internal_1995} assume two kinds of internal models: a forward model (or simulator) that predicts the sensory consequences of motor commands from efference copies of the issued motor commands, and an inverse model (or controller) that calculates the feedforward motor commands from the desired sensory states \citep{gentsch_towards_2016, loevenbruck_cognitive_2018}. Notice that in the following, we do not distinguish further between simulation and emulation, and we use these terms interchangeably to designate the mechanism through which sensory consequences of motor commands (or copy thereof) are computed by forward internal models.

% By building upon models of speech motor control (e.g., Houde \& Nagarajan, 2011; Wolpert et al., 1995), a recent model describes wilful (voluntary) expanded covert speech as "multi-modal acts with multi-sensory percepts stemming from coarse multi-sensory goals" \cite{loevenbruck_cognitive_2018}. In other words, this model considers the auditory and kinaesthetic sensations perceived during covert speech to be the predicted sensory consequences of inhibited speech motor acts, emulated by internal forward models that use the efference copies issued from an inverse model (this proposal shares some similarities with the emulation model of motor imagery, Grush, 2004) \cite{grush_emulation_2004}. This model is supported by several empirical results providing both behavioural and neurophysiological evidence for the efference copies issued during inner speech production (e.g., Scott, 2013; Whitford et al., 2017)...

The motor simulation theory \citep{jeannerod_origin_2006, jeannerod_representing_1994, jeannerod_neural_2001} of motor imagery postulates the existence of a continuum between the covert and the overt execution of an action and that action representations can operate off-line via a simulation mechanism. In short, covert actions would rely on the same set of mechanisms as the overt actions they simulate, except that execution is inhibited \citep{oshea_does_2017}. In this perspective, we examine some theoretical and experimental consequences that emerge when considering covert speech as inhibited overt speech. We explore two main themes. First, we discuss the under-explored role of inhibitory mechanisms during covert speech production and their plausible neural implementation. Second, we consider how different forms of covert speech may involve inhibitory mechanisms differently. By bridging together recent results from the motor imagery, motor inhibition, and covert speech literature, we highlight some novel and possibly fruitful lines of research.

\section{Covert speech production as inhibited overt speech production}

The proposal that overt and covert actions share common neural circuits is faced with a serious problem. If the neural circuits used for the control of overt actions are reused for covert actions, how can covert actions not lead to execution? This puzzle, coined as \textit{the problem of inhibition} by \cite{jeannerod_neural_2001}, can be rephrased as follows: given the putative role of the motor system in providing the multisensory content of motor imagery, how is it possible for motor imagery not to lead to motor execution? In the following, we briefly summarise the evidence in favour of the view of covert speech as inhibited overt speech. Then, we outline some working hypotheses regarding the cognitive and neural implementation of these inhibitory mechanisms. Finally, we relate this view to the presumed progressive internalisation of speech during childhood.

\subsection{Cognitive mechanisms and neural implementation}

First and foremost, we need to make a distinction between at least two different types of inhibition: the inhibition of physical response (e.g., planning to raise the arm and finally not doing it), and the idea of cognitive inhibition, defined as "the stopping or overriding of a mental process, in whole or in part, with or without intention" \citep{gorfein_concept_2007}. Here, we are concerned with the former, which can be defined broadly as the withholding, suppression, or overriding of an inappropriate, prepotent, or unwanted motor response \citep{aron_neural_2007, oshea_go_2018}. To be more precise, \cite{ridderinkhof_dont_2014} described the concept of response inhibition on three continuous dimensions: intentionality, premeditation, specificity. Inhibition can be employed with more or less intentionality (intentional vs. reactive inhibition), it can planned ahead or employed in the moment (early vs. late inhibition), and it can be more or less specifically (global inhibition versus effector-specific or action-specific inhibition).

Crucially, we hypothesise that covert speech involves an intentional (we know we want to produce these actions covertly rather than overtly) but automatic (we do not explicitly think about not producing movements), proactive (i.e., planned ahead), and possibly global form of response inhibition. Notice however that the type of motor inhibition that may be at play during motor imagery is still different from the "proactive inhibition" in the motor inhibition literature, because while doing motor imagery, participants are not explicitly asked not to produce an action \citep{guillot_imagining_2012}. Instead, they are asked to produce the action \textit{covertly}, which (only implicitly) implies that it should not be executed overtly. Using an action mode (overt vs. covert) switching paradigm, \cite{rieger_inhibition_2017} and \cite{bart_inhibitory_2021} have showed that the motor imagery of hand movements is accompanied by both global and effector-specific inhibition \citep[these results were also replicated in][]{scheil_motor_2018}.

% further confirmed that both global and effector-specific inhibition may be at play during imagined (hand) movements and suggested that these two types of inhibition may work together and compensate for each other... \cite{schwoebel_man_2002}...

After reviewing evidence from electrophysiological, neuroimaging, and clinical studies, \cite{guillot_imagining_2012} suggested three possible routes through which motor commands can be inhibited during motor imagery. First, motor inhibition can be integrated within the representation of the action to be produced internally, therefore no active inhibition would be involved during motor imagery (but see the last paragraph of the current section). Second, cerebral regions such as the supplementary motor area (SMA) \citep{kasess_suppressive_2008} or the right inferior frontal gyrus (rIFG) may progressively weaken the motor commands that are emitted during motor imagery \citep[e.g.,][]{angelini_motor_2015, angelini_proactive_2016}, in addition to modulations of short-interval intracortical inhibition within the primary motor cortex itself \citep{neige_unravelling_2020}. More precisely, the pre-SMA and the rIFG may work together to send a stop signal in order to intercept the action process via the basal ganglia (subthalamic nucleus, STN), suppressing the output from the basal ganglia and resulting in inhibitory effects on the primary motor cortex \citep{aron_reactive_2011} (cf. Figure \ref{fig:2}).\footnote{\cite{diesburg_pause-then-cancel_2021} proposed a functional dissociation between the preSMA and the rIFC in a two-stage "pause-then-cancel" model of action inhibition. In their model, they suggest that the preSMA would be involved in (globally) pausing the action whereas the rIFC would be involved in selectively cancelling (resetting) motor commands. Whether this functional dissociation is preserved in motor imagery remains an open question.} Third, downstream regions in the cerebellum \citep[e.g.,][]{lotze_activation_1999}, in the brainstem \citep[e.g.,][]{jeannerod_neural_2001, jeannerod_motor_2006}, or at the spinal level may contribute to motor inhibition at a later stage.

The cause role of inhibitory mechanisms during motor imagery and covert speech could be assessed in several ways. First, it could be assessed by experimentally manipulating the activity of the inhibitory network responsible for preventing execution during motor imagery. For instance, TMS could be used to "brake" these inhibitory mechanisms and "force" or facilitate execution during motor imagery \citep[e.g.,][]{angelini_motor_2015, angelini_proactive_2016}. Second, the role of inhibitory mechanisms during covert verbal actions could be examined in a population with well-identified inhibitory deficits. For instance, Tourette syndrome is a childhood-onset neurological disorder affecting approximately 1\% of children and characterised by chronic motor and phonic tics \citep{jackson_inhibition_2015}. Verbal tics can consist of repeating sounds, words, or utterances (palilalia), producing inappropriate or obscene utterances (coprolalia), or the repetition of another’s words (echolalia). \cite{ganos_action_2014} have showed that the behavioural performance during a stop-signal reaction-time task was similar across Tourette patients and healthy controls. In their review, \cite{jackson_inhibition_2015} suggested that increased control over motor outputs is brought about by local increases in GABAergic "tonic" inhibition within regions such as the SMA, leading to localised reductions in the gain of motor excitability. For these reasons, comparing the neural implementation of inhibitory mechanisms during covert verbal actions in patients with TS and healthy controls may shed light on the role and flexibility of these mechanisms and may lead to applied outcomes in the care of motor and verbal tics.

Whether covert speech production is accompanied by the emission of motor commands that are subsequently inhibited has been the matter of debate. It has been suggested that the primary motor cortex could be "bypassed" during covert speech  \citep[e.g.,][]{tian_mental_2012, tian_effect_2013, tian_mental_2016} or that motor commands would be emitted but subsequently inhibited by frontal regions \citep[e.g.,][]{loevenbruck_cognitive_2018}. However, stating that motor imagery only involve subthreshold activity (and therefore is not accompanied by the emission of motor commands that are inhibited) simply shifts the problem from "how and where motor commands are subsequently inhibited" to "how and where the magnitude of activity in the motor system is planned/regulated/monitored" \citep[see also][]{scheil_motor_2018}. In other words, we still need to explain how (in a mechanistic and/or developmental way) this activity is maintained at a subthreshold level. What cognitive and neural mechanisms operate to maintain this activity under the threshold? In this section, we provided empirical arguments in favour of the "active inhibition hypothesis". Proponents of the "subliminal level hypothesis" need to clarify how this activity is maintained at a subthreshold level during motor imagery, thus preventing execution.

\subsection{Covert speech development: Learning not to produce speech}

\cite{watson_psychology_1919} suggested that thought (i.e., covert speech, in his terminology) was rooted in (overt) speech, that is, that covert speech matures from overt speech. \cite{vygotsky_thought_1934} then prominently formulated the idea that covert speech originates in overt speech, and more specifically, in private egocentric speech, that is, self-addressed overt speech in childhood. Vygotsky observed, as Piaget before, that egocentric or "private" speech tends to be internalised during child development. \cite{fernyhough_alien_2004} extended this model with four levels (stages) of internalisation: external dialogue, private speech, expanded inner speech and condensed inner speech. These levels represent stages of development but also defines movements between levels (i.e., how we go from overt to covert speech and conversely). The level at which speech is expressed may depend on inhibitory control applied at different levels in the production flow, such as the formulation or the articulatory planning level \citep{grandchamp_condialint_2019}. Therefore, producing speech covertly crucially depends on successfully inhibiting speech production at several levels. We formulate the hypothesis that the progressive internalisation of speech during childhood may be related to the development of inhibitory abilities.

This idea could be assessed in several ways. First, the relation between speech internalisation and inhibitory abilities could be assessed during development at the critical ages (i.e., between 6 and 8 years). We would expect the ability to imagine actions and speech specifically to be positively correlated with inhibitory skills at this age. \cite{wang_relationship_2021} provided correlational evidence that motor imagery and motor inhibition performance improve together between 7 and 11 years old, and that these two abilities correlated at 7 years old (but did not correlate at 11 years old). This suggests that inhibitory control may play a special role when speech is being internalised, but its role may weaken with expertise. This is consistent with results from training studies suggesting that we rely more and more on domain-general (e.g., memory-based) processes with expertise \citep[e.g.,][]{tarr_mental_1989, jolicoeur_time_1985}.

% This is consistent with results from training studies suggesting that we rely more and more on domain-general (e.g., memory-based) processes with expertise (e.g., Tarr and Pinker, 1989;Jolicoeur, 1985)

Second, the hypothesised co-development of motor imagery and response inhibition abilities could be assessed by examining how novel actions are internalised in adults.\footnote{While keeping in mind the obvious limitation that the child mind is not equivalent to the adult mind, nor is it equivalent to a smaller version of the adult mind. Nevertheless, examining the development of novel imagined actions in adults avoids the contamination of the process of interest (imagined action) by developmental confounds.} For instance, Let’s consider the analogy between speaking and playing a music instrument (e.g., playing the piano). Essentially, learning to play the piano can be said to consist in learning to produce and coordinate complex and fine-grained motor sequences that in turn produce multisensory (e.g., kinaesthetic, auditory, visual) feedback to the agent producing of the action. Therefore, it seems that (from a certain level of analysis), the act of speech can be paralleled with the act of playing an instrument in that it consists in the coordination of complex movements that result in some modifications of the environment, that in turn generate sensory feedbacks (e.g., kinaesthetic, auditory) for the agent. This analogy suggests that we might be able to study the development of internal models responsible for the sensory experience accompanying imagined actions in the adult mind (e.g., when an individual is learning either a novel music instrument or a new language with speech sounds that are not present in his/her native language). By examining the development of novel imagined actions in the adult mind and by using motor interference (e.g., articulatory suppression) procedures, we might gain new insights about the internalisation of speech during childhood.

\section{The interplay between motor control and cognition}

Where does the sensory content of covert verbal actions come from? How does motor learning shape these sensory percepts? In other words, how do the motor and perceptual system interact during covert speech? In the first subsection, we discuss two mechanisms that may be responsible for providing the sensory content of covert speech and their respective involvement in different forms of covert speech. In the second subsection, we discuss an exciting avenue for investigating the influence of motor learning on the sensory content of covert speech, namely, the imagery of totally novel (i.e., never executed) speech acts.

\subsection{Where does this sensory content come from?}

The production of covert speech is often (although not always and not for everyone) accompanied by the feeling of hearing speech \cite{hurlburt_investigating_2011}. In other words, covert speech is accompanied by a sensory experience that "feels like" hearing speech. In this section, we discuss the dual stream prediction model \citep{tian_mental_2012, tian_effect_2013, tian_mental_2016}, a theoretical account of the mechanisms leading to the generation of such auditory percepts during covert speech production.

% while noticing that these mechanisms could be extended to other verbal actions as well.

The dual stream prediction model \citep{tian_mental_2012, tian_effect_2013, tian_mental_2016} describes two neural pathways that may provide the auditory content of covert speech. First, the simulation-estimation prediction stream implements a motor-to-sensory transformation via motor simulation, that is, by simulating speech movements and the perceptual changes that would be associated with these movements \citep[see also][for a similar proposal]{loevenbruck_cognitive_2018}. This stream includes cerebral areas involved in speech motor preparation such as the supplementary motor area, the inferior frontal gyrus, the premotor cortex and the insula, as well as cerebral areas involved in somatosensory estimation and perception such as primary and secondary somatosensory regions, the parietal operculum, and the supramarginal gyrus \citep{tian_mental_2016}. Second, the memory-retrieval prediction stream may be used to provide auditory percepts by "reconstructing stored perceptual information in modality-specific cortices" \citep{tian_mental_2016}. In other words, sensory percepts would be generated by activating sensory-specific cortices from object properties stored in long-term memory. This mechanism provides sensory percepts without the need for computing the predicted sensory consequences of (non executed) motor commands. This stream includes semantic networks, including frontal, temporal, and parietal regions \citep[for more details, see][]{tian_mental_2016}.

%\footnote{This distinction shares similarities with the distinction between the prediction-by-simulation or prediction-by-association mechanisms in speech production and comprehension suggested by \cite{pickering_integrated_2013}.}

The balance between these two mechanisms (i.e., simulation vs. association) may depend on the precise instructions given to the participants, which may clue them to produce different forms of covert speech. For instance, either one of these two streams may be recruited preferentially depending on whether participants are instructed to "imagine speaking" or to "imagine hearing" \citep[see also the distinction between the "inner ear" and the "inner voice", e.g.,][]{smith_subvocalization_1992}. In line with this hypothesis, \cite{tian_mental_2016} have shown that imagined speaking more strongly recruits cerebral regions in the simulation stream than imagined hearing, which more strongly recruits cerebral regions in the association stream. Moreover, \cite{ma_distinct_2019} have shown that inner speaking and inner hearing have distinct MEG correlates and effects on a subsequent phonetic (discriminating /ba/ vs. /da/) categorisation task.

In the absence of non-ambiguous instructions, and in line with \cite{tian_mental_2012}, we suggest that the balance between these two mechanisms may also depend on situational (e.g., surrounding noise) and intrinsic (e.g., expertise) characteristics. In addition to \cite{tian_mental_2012}, we suggest that a common currency determining the use of either one of these mechanisms is the computational cost of (or equivalently, the computational resources available for) each alternative. To clarify, we borrow the concept of memoisation as applied to cognition and mental imagery by \cite{dasgupta_memory_2021} (cf. Box \ref{memoisation}). In their view, memory can be considered as a computational resource to facilitate computational reuse through memoisation. In the context of motor and speech imagery, memoisation can be seen in the greater reliance on memory (vs. motor processes) in the course of learning, where motor-to-sensory mappings are (party or completely) retrieved from memory, instead of being computed again every time.

% \vspace{5mm}

\begin{mybox}[label = memoisation]{Memoisation}

Memoisation is a programming technique used to speed-up algorithms or programs. It avoids redundant computation by storing computational results and reusing them later. When calling a function (where a function can be a motor primitive), the function call is intercepted by a \textit{memoiser} that inspects the previous calls of a function and its outputs. If a function has already been called with the same input, then the previously computed output is retrieved and reused.

\end{mybox}

In other words, situational (extrinsic) and individual (intrinsic) characteristics jointly determine the computational cost of (or equivalently, the available computational resources for) the task, which in turn determines the balance between the simulation and association mechanisms. For instance, we hypothesise that novel and/or difficult tasks (which are both computationally more expensive, ceteris paribus) may rely more on the simulation mechanism, whereas well known and/or easy tasks may rely more on associative mechanisms. This idea is supported by several studies showing a greater increase in facial EMG activity during the reading of difficult text or while performing difficult mental arithmetic tasks as compared to easier tasks \citep[e.g.,][]{faaborg-andersen_electromyography_1958, sokolov_inner_1972}, suggesting a greater involvement of the speech motor system \citep[and/or a lesser involvement of inhibitory mechanisms, see also the discussion in][]{nalborczyk_understanding_2019-1, nalborczyk_re-analysing_2020}. This is congruent with the greater reliance on associative mechanisms with greater expertise, as discussed previously. 

% This is consistent with results from training studies suggesting that we rely more and more on domain-general (e.g., memory-based) processes with expertise \citep[e.g.,][]{tarr_mental_1989, jolicoeur_time_1985}.

\subsection{The covert production of novel verbal actions}

If, as suggested in the previous section, the sensory content of covert speech is, at least in some situations, provided by a motor simulation process, how can we imagine totally novel actions? Broadly speaking, the simulationnist framework suggests that the multisensory experience of motor imagery results from the simulation of the corresponding motor action, reusing internal models developed for the control of overt actions \citep[e.g.,][]{jeannerod_representing_1994}. In other words, internal models developed for the control of actions would then be used to simulate the sensory consequences of covert actions. Crucially, these internal models are generative models, which means that they can be used to compute or predict the sensory consequences of any action that can be parsed as a \citep[possibly supramodal, see][]{loevenbruck_cognitive_2018} goal to an inverse internal model. In other words, this framework makes it possible for novel actions to be simulated internally.\footnote{\cite{mulder_role_2004} assessed whether the mental practice of a totally novel movement (abduction of the big toe of the dominant right foot) resulted in better overt execution over the course of several weeks of practice. They observed that although mental practise did improve execution in the group of participants that already had some experience with the task at baseline, mental practise did not lead to statistically significant improvements in execution in the group of participants that had zero experience with the task at baseline. However, a Bayesian reanalysis of their results suggests that the evidence in favour of the null hypothesis was rather inconclusive ($\text{BF}_{01} = 1.92$, using a weakly informative Cauchy prior on the standardised effect size for the alternative hypothesis, with $r = 1$).}

A general prediction of this framework is that the imagination of novel actions, and more precisely, the predicted sensory consequences of novel actions, may be "biased" or "constrained" by the repertoire of already-known actions. For instance, when asked to imagine playing table tennis, a tennis player may overestimate movements' amplitude and duration. Translated to speech production, this prediction could be assessed by asking participants to imagine a novel verbal action and subsequently checking whether this improves the execution of an "intermediate" novel action in the articulatory space. The same logic could be applied to the covert production of speech with someone else's voice. How can we imagine speaking with the voice of our relatives? If it is based on simulators, then the covert production of speech with someone else's voice should possess sensory properties that are somehow "biased" or "constrained" by the properties of our own sensory and motor systems. In other words, the subjective auditory sensation of covert speech with someone else's voice should be situated in a somehow intermediate (i.e., between the acoustic properties of our own voice and the acoustic properties of someone else voice) location in the auditory space.

This prediction could be assessed experimentally by using the sensory attenuation effect observed during covert speech production. Auditory stimuli elicit an electrophysiological brain response with a characteristic N1 component. This N1 component has been shown to be reduced when the presentation of an auditory stimulus is synchronised with the self-production of a phonetically-matched vocalisation, an effect called N1-suppression. More interestingly, \cite{whitford_neurophysiological_2017} have shown that the covert production of a phoneme also resulted in sensory attenuation, as assessed by N1-suppression, when the content of the imagined phoneme matched the content of the audible phoneme. Therefore, we could describe the subjective auditory sensation of covertly speaking with someone else’s voice by manipulating the acoustic properties of the auditory stimulus that is presented to the participant to find the stimulus that produces the greatest sensory attenuation, as assessed by N1-suppression. In other words, sensory attenuation should be the strongest when it matches the auditory representation of the speaker. If the auditory percepts generated when covertly speaking with the voice of someone else is "constrained" by our own sensory and motor systems, then this point of greatest sensory attenuation should occur in an intermediate location in the feature space.

\section{Conclusions}

% Summary of the main proposition (i.e., considering covert verbal actions as inhibited verbal actions) and its consequences on the relation between perception and action + theoretical and experimental perspectives.

To sum up, we explored some of the theoretical and experimental consequences that emerge when considering covert speech production as inhibited overt speech production. This led us to a discussion of two main themes. First, we asked how it is possible for covert speech production not to lead to overt speech production. Second, we discussed two mechanisms that may be used to provide the sensory content of covert speech, and how their respective involvement may be constrained by the available computational resources. To sketch an answer to these questions, we connected results from the motor imagery, motor inhibition, and covert speech research traditions.

Regarding the role of general-purpose inhibitory mechanisms during the production of covert speech, we suggested that these may be similar to the inhibitory network responsible for proactive response inhibition and we summarised some propositions from this literature. We related the development of response inhibition abilities in childhood development with the purported internalisation of private speech around the same period. From the response inhibition perspective, the internalisation of speech from overt to covert speech may essentially be considered as "learning not to produce speech".

Regarding the neural origin of the sensory experience of covert speech, we discussed the dual stream prediction model \citep{tian_mental_2012, tian_effect_2013, tian_mental_2016}, which suggests that these sensory percepts may be provided either by a motor-simulation-based process or by a memory-based process. We suggested that the balance between these two mechanisms, in addition to the task instructions (which may prompt different forms of covert speech), may also be determined by the available computational resources for (or equivalently, the computational cost of) the task, with novel or more difficult tasks being more costly. More precisely, novel or more difficult tasks are expected to rely more on the motor-simulation mechanisms whereas well-known and/or easy tasks may rely more on a "memoised version" of the motor simulation: the memory-retrieval prediction stream. Whereas the former mechanism should involve active inhibitory mechanisms, the latter should not.

Taken together, these propositions pave the way for several lines of research that should consolidate our understanding of the relations between the speech production and perception systems. Several outstanding questions remain. Amongst others...

\section*{Conflict of Interest Statement}

% All financial, commercial or other relationships that might be perceived by the academic community as representing a potential conflict of interest must be disclosed. If no such relationship exists, authors will be asked to confirm the following statement:

The authors declare that the research was conducted in the absence of any commercial or financial relationships that could be construed as a potential conflict of interest.

\section*{Author Contributions}

% \color{blue}

% The Author Contributions section is mandatory for all articles, including articles by sole authors. If an appropriate statement is not provided on submission, a standard one will be inserted during the production process. The Author Contributions statement must describe the contributions of individual authors referred to by their initials and, in doing so, all authors agree to be accountable for the content of the work. Please see  \href{http://home.frontiersin.org/about/author-guidelines#AuthorandContributors}{here} for full authorship criteria.

% \color{black}

Conceptualisation: all authors; Funding acquisition: LN; Supervision: UD, ML, AG, FXA; Writing - original draft: LN; Writing - review and editing: all authors.

\section*{Funding}

% Details of all funding sources should be provided, including grant numbers if applicable. Please ensure to add all necessary funding information, as after publication this is no longer possible.

This work, carried out within the Institut Convergence ILCB (ANR-16-CONV-0002), has benefited from support from the French government, managed by the French National Agency for Research (ANR) and the Excellence Initiative of Aix-Marseille University (A*MIDEX).

\section*{Acknowledgements}

None.

% \section*{Supplemental Data}

% \href{http://home.frontiersin.org/about/author-guidelines#SupplementaryMaterial}{Supplementary Material} should be uploaded separately on submission, if there are Supplementary Figures, please include the caption in the same file as the figure. LaTeX Supplementary Material templates can be found in the Frontiers LaTeX folder.

\section*{Data Availability Statement}

No novel data were used in this paper. However, the source (\LaTeX) code is available at \url{https://osf.io/dsfgb/}.

% Please see the availability of data guidelines for more information, at https://www.frontiersin.org/about/author-guidelines#AvailabilityofData

\bibliographystyle{template/frontiersinSCNS_ENG_HUMS} % for Science, Engineering and Humanities and Social Sciences articles, for Humanities and Social Sciences articles please include page numbers in the in-text citations
%\bibliographystyle{frontiersinHLTH&FPHY} % for Health, Physics and Mathematics articles
% \bibliography{test}
\bibliography{bibliography/covert_verbal_actions}

%%% Make sure to upload the bib file along with the tex file and PDF
%%% Please see the test.bib file for some examples of references

\section*{Figure captions}

%%% Please be aware that for original research articles we only permit a combined number of 15 figures and tables, one figure with multiple subfigures will count as only one figure.
%%% Use this if adding the figures directly in the mansucript, if so, please remember to also upload the files when submitting your article
%%% There is no need for adding the file termination, as long as you indicate where the file is saved. In the examples below the files (logo1.eps and logos.eps) are in the Frontiers LaTeX folder
%%% If using *.tif files convert them to .jpg or .png
%%% NB logo1.eps is required in the path in order to correctly compile front page header %%%

% \begin{figure}[ht] % float was h! initially
% \begin{center}
% \includegraphics[width=0.75\textwidth]{figures/simulation_association.png} % This is a *.eps file
% \end{center}
% \caption{High-level depiction of the prediction-by-simulation (in blue) and prediction-by-association (in orange) mechanisms. The balance (weighting) between these two mechanisms during covert verbal actions depends on both task demands and "computational cost" (cf. text for more details), which are jointly determined by internal (individual) and external (situational) characteristics (e.g., expertise or (equivalently) task difficulty, noise, feedback perturbation).}\label{fig:1}
% \end{figure}

\begin{figure}[ht] % float was h! initially
\begin{center}
\includegraphics[width=0.75\textwidth]{figures/inhibitory_triangle.png} % This is a *.eps file
\end{center}
\caption{Plausible implementation of the cortical and subcortical inhibitory mechanisms responsible for the "proactive" (but implicit) response inhibition at play during covert speech production. The preSMA, right pIFC, and STN together form an \textit{inhibitory network} known as the \textit{inhibitory triangle}, which may be responsible for braking motor commands during covert speech production. Figure created with BioRender.com. FIGURE IN PROGRESS.}\label{fig:2}
\end{figure}

%%% If you are submitting a figure with subfigures please combine these into one image file with part labels integrated.
%%% If you don't add the figures in the LaTeX files, please upload them when submitting the article.
%%% Frontiers will add the figures at the end of the provisional pdf automatically
%%% The use of LaTeX coding to draw Diagrams/Figures/Structures should be avoided. They should be external callouts including graphics.

\end{document}
